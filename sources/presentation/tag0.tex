\section{Organisation}

\begin{frame}{Disclaimer - Quellen dieser Unterlagen}
  \begin{alertblock}{Wichtiger Hinweis zu den Inhalten}
    \begin{itemize}
      \item Diese Materialien basieren direkt auf den Vorlesungsunterlagen von Jens Bendisposto
      \item Viele Definitionen, Beispiele und Textpassagen sind \textbf{1:1 übernommen}
      \item Ergänzende Inhalte stammen von der offiziellen Java-Dokumentationen und \textbf{Paul C. Dötsch}
      \item \textbf{Kein Plagiat beabsichtigt} - sondern bewusste Verwendung der Originalmaterialien
    \end{itemize}
  \end{alertblock}

  \begin{exampleblock}{Begründung}
    \begin{itemize}
      \item Konsistenz mit der Vorlesung und einheitliche Terminologie
      \item Bewährte Erklärungen und direkt für die Klausurvorbereitung nutzen
    \end{itemize}
  \end{exampleblock}

  \begin{center}
    {\footnotesize \textit{Alle Rechte an den verwendeten Materialien liegen bei den jeweiligen Autoren. \\
    Diese Zusammenstellung dient ausschließlich Lehr- und Lernzwecken.}}
  \end{center}
\end{frame}

\begin{frame}{Wie für die Klausur lernen (offizielle Version)}
  \begin{alertblock}{Nur Altklausuren lernen = Schiefgehen!}
    \begin{itemize}
      \item Altklausuren sind \textbf{kein Ersatz} für intensive Mitarbeit im Semester
      \item Klausuren prüfen \textbf{tieferes Verständnis} ab, nicht nur Auswendiglernen
      \item \textbf{Alle Wochenblätter sind relevant}, auch die letzten Blätter
      \item Aufgaben sind \textbf{keine umformulierten Übungsaufgaben}
    \end{itemize}
  \end{alertblock}

  \begin{exampleblock}{Richtige Lernstrategie}
    \begin{itemize}
      \item Von jedem Wochenblatt \textbf{Wichtige Begriffe} erklären können \& die \textbf{Lernziele} beachten
      \item \textbf{Alle Aufgaben selbstständig} bearbeiten
      \item \textbf{Anderen den Stoff erklären} (Lerngruppen!)
      \item Verstehen \textbf{warum} Regeln existieren und \textbf{wann} sie gebrochen werden dürfen
    \end{itemize}
  \end{exampleblock}
\end{frame}

\begin{frame}{Nachklausurtutorium Programmierpraktikum 1}
  \textbf{Zeitraum:} Montag 15.09.2025 bis Freitag 19.09.2025

  \textbf{Zeit:} Täglich 08:30 - 12:30 Uhr

  \textbf{Ort:} Hörsaal 5M

  \vspace{0.5cm}

  \begin{exampleblock}{Wichtige Links}
    \begin{itemize}
      \item \textbf{GitHub für dieses Tutorium} (hier findet ihr auch diese Präsentation): \url{https://github.com/hhu-propra1-ss25/nachklausurtutorium}
      \item \textbf{GitHub Organisation:} \url{https://github.com/hhu-propra1-ss25/Organisation}
    \end{itemize}
  \end{exampleblock}
\end{frame}

\begin{frame}{Tagesplanung - Überblick}
  \begin{columns}[T]
    \column{0.5\textwidth}
    \textbf{Tag 1 - Grundlagen \& Werkzeuge}
    \begin{itemize}
      \item Java-Basics \& Collections
      \item Generics
      \item Funktionale Programmierung
      \item Streams
      \item Gradle \& Git
    \end{itemize}

    \textbf{Tag 2 - Testing \& Codequalität}
    \begin{itemize}
      \item Testing Basics \& TDD
      \item Code Smells im Kleinen
      \item Wartbarkeit
    \end{itemize}

    \column{0.5\textwidth}
    \textbf{Tag 3 - Architektur \& Prinzipien}
    \begin{itemize}
      \item SOLID-Prinzipien
      \item Vererbung \& Polymorphismus
      \item Code Smells im Großen
    \end{itemize}

    \textbf{Tag 4 - Fortgeschrittene Themen}
    \begin{itemize}
      \item Mocking \& Test-Doubles
      \item Spring Framework
      \item Git im Detail
    \end{itemize}

    \textbf{Tag 5 - Klausurvorbereitung}
    \begin{itemize}
      \item Altklausuren lösen
    \end{itemize}
  \end{columns}
\end{frame}

\begin{frame}
  \thispagestyle{empty}
  \frametitle{Was diese Folien sind}
  \begin{columns}[T]
    \column{0.5\textwidth}
    \textbf{Diese Präsentation IST:}
    \begin{itemize}
      \item \textbf{Zentrale Lernhilfe} - Alles Wichtige an einem Ort
      \item \textbf{Nachschlagewerk} - Ctrl+F für Themen
      \item \textbf{Spickzettel-freundlich} - Einfach kopieren
      \item \textbf{Klausurvorbereitung} - Was ihr wahrscheinlich braucht
      \item \textbf{Meine Interpretation} der Modulinhalte
      \item Was ich mir mir als Student gewünscht hätte!
    \end{itemize}

    \column{0.5\textwidth}
    \textbf{Diese Präsentation ist NICHT:}
    \begin{itemize}
      \item \textbf{Offiziell} - nicht von Jens Bendisposto oder Markus Brenneis
      \item \textbf{Fehlerfrei} - kann kleine Fehler enthalten
      \item \textbf{Vollständig} - ersetzt nicht die Vorlesung
      \item \textbf{Einzige Quelle} - immer gegenchecken!
      \item \textbf{Garantie} für Klausurerfolg
    \end{itemize}
  \end{columns}
\end{frame}

\begin{frame}
  \thispagestyle{empty}
  \frametitle{Wichtiger Hinweis}
  \begin{alertblock}{Disclaimer}
    \begin{itemize}
      \item Diese Präsentation ist \textbf{meine persönliche Interpretation} der Modulinhalte
      \item Sie ist \textbf{nicht offiziell} und kann Fehler enthalten
      \item \textbf{Immer gegen offizielle Quellen prüfen}:
      \begin{itemize}
        \item Vorlesungswebsite von Jens Bendisposto
        \item Online-Dokumentation (Java Docs, etc.)
        \item Offizielle Referenzen aus der Vorlesung
      \end{itemize}
      \item Bei Unsicherheiten: Fragen stellen!
      \item Mein Ziel: Euch beim Bestehen helfen, nicht die Vorlesung ersetzen
    \end{itemize}
  \end{alertblock}

  \begin{center}
    {\small \textit{Diese Folien dienen als Lernhilfe!}}
  \end{center}
\end{frame}
