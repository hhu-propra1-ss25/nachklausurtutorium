\section{Tag 5 - Klausurvorbereitung}

\subsection{Klausurvorbereitung}

\begin{frame}{Nachklausur 2025 - Organisatorisches}
  \begin{exampleblock}{Grunddaten}
    \begin{itemize}
      \item \textbf{Dauer:} 90 Minuten
      \item \textbf{Datum:} Montag, 22.09.2025
      \item \textbf{Beginn:} 9:00 Uhr
      \item \textbf{Einlass:} ab ca. 8:45 Uhr
      \item \textbf{Hörsaal:} Per Studierendenportal oder Email bekanntgegeben
      \item \textbf{Problem?} Sofort melden unter: propra@cs.hhu.de
    \end{itemize}
  \end{exampleblock}

  \begin{alertblock}{Teilnahme-Voraussetzungen}
    \begin{itemize}
      \item Nur \textbf{angemeldete Personen} mit Klausurzulassung
      \item \textbf{Amtlicher Lichtbildausweis} (Personalausweis, Reisepass, dgti+Ausweis) erforderlich
    \end{itemize}
  \end{alertblock}
\end{frame}

\begin{frame}{Hilfsmittel \& Vorbereitung}
  \begin{exampleblock}{Erlaubte Hilfsmittel}
    \begin{itemize}
      \item \textbf{Pflicht:} Eine beidseitig \textbf{handschriftlich} beschriebene A4-Seite
      \item \textbf{Optional:} Wörterbuch der deutschen Sprache (bei Einlass vorzeigen)
      \item Elektronische Geräte \textbf{ausschalten} und in Tasche packen
    \end{itemize}
  \end{exampleblock}

  \begin{alertblock}{Wichtige Regeln}
    \begin{itemize}
      \item Spickzettel \textbf{nicht kopiert/gedruckt} - nur handschriftlich!
      \item Eigene Schmierblätter \textbf{nicht erlaubt} - bei Aufsicht anfordern
      \item Toilettengang mit elektronischen Geräten = Täuschungsversuch
      \item Frühzeitiges Anfangen/Blättern = Täuschungsversuch
    \end{itemize}
  \end{alertblock}
\end{frame}

\begin{frame}{Tiefe des Stoffs - Beispiel}
  \begin{columns}[T]
    \column{0.5\textwidth}
    \textbf{Oberflächliche Antwort:}
    \begin{alertblock}{Schlecht}
      „Der Code ist schlecht wartbar, weil er das Gesetz von Demeter verletzt"
    \end{alertblock}

    \column{0.5\textwidth}
    \textbf{Tiefe Antwort:}
    \begin{exampleblock}{Gut}
      Erklären Sie:
      \begin{itemize}
        \item \textbf{Unter welchen Umständen} sich die Verletzung negativ auswirkt
        \item \textbf{Konkrete Stellen} im Code
        \item \textbf{Warum} das die Wartbarkeit beeinflusst
      \end{itemize}
    \end{exampleblock}
  \end{columns}

  \vspace{0.5cm}
  \begin{center}
    \textbf{Wichtig:} Die Tiefe der Wochentests ist \textbf{NICHT} repräsentativ für die Klausur!
  \end{center}
\end{frame}

\begin{frame}{Klausuraufbau \& Strategie}
  \begin{columns}[T]
    \column{0.5\textwidth}
    \textbf{Aufgaben überspringen:}
    \begin{itemize}
      \item Überblick über \textbf{alle Aufgaben} verschaffen
      \item Mit den \textbf{einfachsten} Aufgaben beginnen
      \item Sich \textbf{nicht verbeißen} - weiter zur nächsten Aufgabe
      \item Zeit nicht vergeuden
    \end{itemize}

    \textbf{Code schreiben:}
    \begin{itemize}
      \item Maximal \textbf{10-15 Zeilen} pro Teilaufgabe
      \item Statische Imports werden \textbf{nicht verlangt}
      \item Kleine Methodennamen-Fehler sind \textbf{okay}
      \item Muss \textbf{erkennbar} sein was gemeint ist
    \end{itemize}

    \column{0.5\textwidth}
    \textbf{Präzise antworten:}
    \begin{alertblock}{Punktabzug für:}
      \begin{itemize}
        \item Falsche Anteile neben korrekter Antwort
        \item Anteile, die nichts mit der Frage zu tun haben
        \item Auflistung nicht vorhandener Code Smells
        \item Unschlüssige Begründungen
      \end{itemize}
    \end{alertblock}

    \textbf{Punkte-Vergabe:}
    \begin{itemize}
      \item Korreliert \textbf{nicht} mit Arbeitsaufwand
      \item Bezieht sich auf \textbf{fachliche Aspekte}
      \item Jeder Aspekt wird einzeln bewertet
    \end{itemize}
  \end{columns}
\end{frame}

\begin{frame}{Tipps zur Vorbereitung}
  \begin{columns}[T]
    \column{0.5\textwidth}
    \textbf{Aktive Vorbereitung:}
    \begin{itemize}
      \item \textbf{Wichtige Begriffe} von jedem Wochenblatt erklären können
      \item \textbf{Alle Aufgaben selbstständig} bearbeiten
      \item Lösungen \textbf{selbst erarbeiten}, dann mit Musterlösungen vergleichen
      \item \textbf{Spickzettel selbst} zusammenstellen
      \item \textbf{Auf Papier üben} (besonders Programmieraufgaben)
    \end{itemize}

    \column{0.5\textwidth}
    \textbf{Soziales Lernen:}
    \begin{itemize}
      \item \textbf{Anderen den Stoff erklären}
      \item In \textbf{Lerngruppen diskutieren}
      \item Sich \textbf{gegenseitig Aufgaben stellen}
      \item \textbf{Lücken gemeinsam} identifizieren
    \end{itemize}

    \textbf{Orientierung:}
    \begin{itemize}
      \item \textbf{Lernziele} auf den Wochenblättern beachten
      \item Klausur prüft, ob Lernziele erreicht wurden
    \end{itemize}
  \end{columns}

  \begin{exampleblock}{Häufige Klausurthemen}
    \textbf{Meist vorhanden:} Streams, Testing, Prinzipien, Code Smells \\
    \textbf{Aber:} Alle anderen Themenbereiche können auch abgefragt werden!
  \end{exampleblock}
\end{frame}

\begin{frame}{Klausur-Ablauf}
  \begin{columns}[T]
    \column{0.5\textwidth}
    \textbf{Vor der Klausur:}
    \begin{itemize}
      \item \textbf{Identitätskontrolle} und Sitzzuweisung
      \item Klausur liegt bereits am Platz aus
      \item \textbf{NICHT} vorzeitig öffnen oder schreiben!
    \end{itemize}

    \textbf{Während der Klausur:}
    \begin{itemize}
      \item Ausweis sichtbar auf dem Tisch legen
      \item \textbf{Nur Verständnisfragen} möglich
      \item Interpretation bei Unsicherheit vermerken
      \item Schmierblätter bei Aufsicht anfordern
    \end{itemize}

    \column{0.5\textwidth}
    \textbf{Ende der Klausur:}
    \begin{itemize}
      \item Bei Zeitende: \textbf{Sofort Stifte hinlegen}
      \item Nichts mehr schreiben
      \item Klausur am Platz liegen lassen
    \end{itemize}

    \textbf{Bei Krankheit:}
    \begin{itemize}
      \item Einfach \textbf{nicht erscheinen} - kein Fehlversuch
      \item Während Klausur: Abbruch explizit melden
      \item \textbf{Nicht} regulär abgeben
      \item Ich wiederhole: \textbf{Niemals unterschreiben!!!!!}
    \end{itemize}
  \end{columns}

  \vspace{0.25cm}

  \begin{exampleblock}{Nach der Klausur}
    Ergebnisse im Studierendenportal + Klausureinsicht mit Anmeldung erforderlich
  \end{exampleblock}
\end{frame}
